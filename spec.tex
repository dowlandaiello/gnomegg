\documentclass{article}

\usepackage[english]{babel}
\usepackage{microtype}
\usepackage{graphicx}
\usepackage{wrapfig}
\usepackage{enumitem}
\usepackage{fancyhdr}
\usepackage[margin=1.0in]{geometry}
\usepackage{qtree}
\usepackage{float}
\usepackage{booktabs}
\usepackage{tabularx}
\usepackage{textcomp}

\begin{document}
\title{Gnomegg: A blazing-fast third-party destiny.gg chat server}
\author{Dowland Aiello}
\date{April 17, 2020}

\maketitle
\tableofcontents
\fancyhf{}

\newpage

\section{Data Schematic}

Gnomegg is entirely backwards-compatible with the destiny.gg chat, and uses
capnproto. Cap'n proto is generally regarded as the successor to Google's
protobuf language, and is both platform and language-agnostic. Capnproto poses
several benefits over protobuf. Namely:

\begin{itemize}
	\item Cap'n Proto serves as both an interchange and in-memory storage
		format---no serialization necessary!
	\item \emph{Promise pipelining}: subsequent messages relying on each other
		can be ``squashed'' into one message
	\item Cap'n Proto data can be transferred via a standard bytestream, and can
		be encrypted in an equally straightforward manner
\end{itemize}

Outlined below are each of the data types represented in Cap'n proto, for use
in gnomegg.

\end{document}
