\documentclass{article}

\usepackage[english]{babel}
\usepackage{microtype}
\usepackage{graphicx}
\usepackage{wrapfig}
\usepackage{enumitem}
\usepackage{fancyhdr}
\usepackage[margin=1.0in]{geometry}
\usepackage{qtree}
\usepackage{float}
\usepackage{booktabs}
\usepackage{tabularx}
\usepackage{textcomp}
\usepackage[T1]{fontenc}

\begin{document}
\title{Gnomegg: A blazing-fast third-party destiny.gg chat server}
\author{Dowland Aiello}
\date{April 17, 2020}

\maketitle
\tableofcontents
\fancyhf{}

\newpage

\section{Data Schematic}

Gnomegg is entirely backwards-compatible with the destiny.gg chat, and uses
capnproto. Cap'n proto is generally regarded as the successor to Google's
protobuf language, and is both platform and language-agnostic. Capnproto poses
several benefits over protobuf. Namely:

\begin{itemize}
	\item Cap'n Proto serves as both an interchange and in-memory storage
		format---no serialization necessary!
	\item \emph{Promise pipelining}: subsequent messages relying on each other
		can be ``squashed'' into one message
	\item Cap'n Proto data can be transferred via a standard bytestream, and can
		be encrypted in an equally straightforward manner
\end{itemize}

Outlined below are each of the data types represented in Cap'n proto, for use
in gnomegg.

\subsection{Events}

In gnomegg, an event represents some action taken by a user that must be
handled by the server, or, in some cases, a response from the server sent to a
chatter. Gnomegg uses all of the events commonplace in the destiny.gg chat server
software, but adds an \textbf{Error} event type. Gnomegg's event types,
alongside their respective fields are defined as such:

\begin{itemize}
	\item issueCommand: a command was issued by a client
		\begin{itemize}
			\item Issuer: the username of the chatter issuing the command
			\item Type: a union with the following possible values representing
				the type of command being issued:
				\begin{itemize}
					\item Message: an object defined as such, representing a
						simple text message sent by a chattter, rendered by the
						client:
						\begin{itemize}
							\item Contents: the contents of the message,
								represented by a UTF-8 encoded string
						\end{itemize}
					\item PrivMessage: an object defined as such, representing
						a message sent only to one chatter, rendered by the client:
						\begin{itemize}
							\item Concerns: the username of the chatter that the
								message should be sent to
							\item Message: a mesage literal as defined above in
								the Message schematic
						\end{itemize}
					\item Mute: an object defined as such, used to mute a user for
						a specified amount of time:
						\begin{itemize}
							\item Concerns: the username of the chatter who will
								be muted
							\item Duration: the number of nanoseconds for which
								the targeted user should be muted, expressed as
								a 64-bit unsigned integer
						\end{itemize}
					\item Unmute: an object containing a string representing
						the username of the user who will be banned
					\item Ban: an object defined as such, used to ban a user
						for a particular stretch of time:
						\begin{itemize}
							\item Concerns: the username of the chatter banned
								as a result of this command's execution
							\item Reason: a string representing a moderator's
								reasoning behind the ban command's issuance
							\item Duration: the number of nanoseconds for which
								the targeted user will be banned from chatting
						\end{itemize}
					\item Unban: an object defined as such, removing a user
						from the ban list
						\begin{itemize}
							\item Concerns: the username of the chatter who will
								be unbanned
						\end{itemize}
					\item Subonly: an object defined as such, making the chat
						sub-only mode:
						\begin{itemize}
							\item On: whether or not the chat should be in sub-
								only mode
						\end{itemize}
					\item Ping: an object defined as such, initiating a pint-
						pong response loop:
						\begin{itemize}
							\item InitiationTimestamp: a timestamp expressed as
								a slice of bytes, representing the time at which
								this command was issued
						\end{itemize}
				\end{itemize}
		\end{itemize}
	\item pong: the server is responding to a client request to ping with a pong
		\begin{itemize}
			\item Timestamp: a slice of bytes representing the time at which
				the request to ping was received
		\end{itemize}
	\item broadcast: the server is broadcasting a message sent by a client to the
		users in the chat session
		\begin{itemize}
			\item Sender: the username of the chatter sending the message
			\item Message: the message sent, represented as a UTF-8 string
		\end{itemize}
	\item error: the server is responding to an individual request with an
		error that occurred at runtime
		\begin{itemize}
			\item Concerns (all | user (username)): the scope of the error---
				should it be sent to all users in chat, or just one user?
			\item Error: a string representing the error that will be
				communicated to connected clients
		\end{itemize}
\end{itemize}

\end{document}
